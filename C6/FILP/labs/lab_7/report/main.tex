\chapter{Цели и задачи работы}
\textbf{Цель работы} --- приобрести навыки органинизации рекурсии в Lisp.
\textbf{Задачи работы} --- изучить способы организации хвостовой, дополняемой, множественной,
взаимной рекурсии и рекурсии более высокого порядка в Lisp.

\chapter{Практические задания}
\section{Задание 1}
Написать хвостовую рекурсивную функцию my-reverse, которая развернет верхний уровень своего списка-аргумента lst.
\begin{lstlisting}
(defun my-reverse_rec (lst new_lst)
  (if (not (eql (cdr lst) nil))
      (my_reverse_rec (cdr lst) (cons (car lst) new_lst))
      (cons (car lst) new_lst)))

(defun my-reverse (lst)
  (my_reverse_rec lst nil))
\end{lstlisting}

\section{Задание 2}
Написать функцию, которая возвращает первый элемент списка-аргумента, который сам является непустым списком.
\begin{lstlisting}
(defun get_first_not_nil (lst)
  (if (not (eql lst nil))
   (if (eql (length (car lst)) 0)
       (get_first_not_nil (cdr lst))
       (caar lst)
       )
   )
  )
\end{lstlisting}

\section{Задание 3}
Написать функцию которая выбирает из заданного списка только те числа, которые больше 1 и меньше 10. (Вариант: между двумя заданными границами).

\begin{lstlisting}
(defun get-interval-rec (lst new-lst a b)
  (if (not (eql lst nil))
      (if (and (< a (car lst)) (> b (car lst)))
          (get-interval-rec (cdr lst) `(,@new-lst ,(car lst)) a b)
          (get-interval-rec (cdr lst) new-lst a b)
          )
      new-lst
      )
  )

(defun get-interval (lst a b)
  (get-interval-rec lst nil a b)
  )
\end{lstlisting}

\section{Задание 4}
Напишите рекурсивную функцию, которая умножает на заданное число-аргумент все числа из заданного списка-аргумента, когда:\\
a) все элементы - числа.\\
б) все элементы - любые объекты.

\section{Задание 5}
Напишите функцию select-between, которая из списка-аргумента, содержащего только числа, выбиает только те, которые расположены между двумя указанными границами-аргументами и возвращает их в виде списка.

\section{Задание 6}
Написать рекурсивную версию с именем rec-add вычисления суммы чисел заданного списка:\\
a) одноуровнего смешанного.\\
b) структурированного.

\section{Задание 7}
Написать рекурсивную версию с именем recnth функции nth.

\section{Задание 8}
Написать рекурсивную функцию allodd, которая возвращает t когда все элементы списка нечетные.

\section{Задание 9}
Написать рекурсивную функцию, которая возвращает первое нечетное число из списка (структурированного), возможно создавая некоторые вспомогательные функции.

\section{Задание 10}
Используя cons-дополняемую рекурсию с одним тестом завершения, написать функцию которая получает как аргумент список чисел, а возвращает список квадратов этих чисел в том же порядке.