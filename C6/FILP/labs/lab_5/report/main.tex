\chapter{Цели и задачи работы}
\textbf{Цель работы} --- приобрести навыки работы с управляющими структурами Lisp.
\textbf{Задачи работы} ---изучить работу функций с произвольным количеством аргументов, функций
разрушающих и неразрушающих структуру исходных аргументов.

\chapter{Теоретические вопросы}
\section{Структуроразрушающие и не разрушающие структуру списка функции}
Функции можно разбить на две группы - разрушающие структуру и не разрушающие структуру. Для работы со списком его необходимо создать, получить доступ и модифицировать. Функции, разрушающие структуру - изменяют структуру списка. Функции не разрушающую структуру - производят какие-то операции без изменения поданного на вход списка.

В функции, разрушающие структуру списка входят - nconc, nreverse
В функции, не разрущающие структуру списка входят - append, reverse, list, cons
\section{Отличие в работе и результат работы функций cons, list, append, nconc}
При работе функции cons создаётся бинарный узел, car которого указывает на первый аргумент, а cdr на второй. В результате работы создаётся новый объект и при этом не разрушаются списки, поданные на вход. 

При работе функции list создаётся новый список из поданных на вход элементом, при этом сами элементы не меняются.

При работе функции append создаётся копия списка, в конец которой добавляется новый элемент. При этом старый список остаётся без изменений.

При работе функции nconc новый список создаётся путём присвоения cdr указателям концов списков начал следуюещго списка. В результате разрушается структура списка, поскольку при попытке получить доступ к старому списку будет выводится новый список начиная с аргумента, с которым был связан символ.

\chapter{Практические задания}
\section{Задание 1}
Написать функцию, которая по своему списку-аргументу lst определяет является ли он палиндромом (то есть равны ли lst и (reverse lst)).

\section{Задание 2}