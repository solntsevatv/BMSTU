\chapter{Сравнительный анализ существующих решений}
Полученное множество систем, позволяющих решить задачу создания трёхмерной среды можно классифицировать по типу системы, по стоимости использования системы, доступности и кроссплатфоменности.

Классификация решений по типу подразумевает разбиение по виду реализации предложенных систем. В данной работе рассматриваются два типа систем --- игровой движок и графический API. Данные решения таже разбиваюся согласно требованиям, которым должно удовлетворять решение, использующееся в разработе трёхмерной виртуальной среды. В частности, этими требованиями являются стоимость использования --- движок или графическое API должен находиться в свободном доступе, либо позволять использовать основные функции в некой бесплатной версии, кроссплатформенность --- чем большее количество платформ покрывает данная система, тем меньше времени требуется на разработку кроссплатформенной версии приложения, доступность ---  использование системой нестандартных языков программирования и систем, не являющихся общепринятым стандартом в своей области, может сильно осложнить разработку трёхмерной визуальной среды, поэтому использование популярных языков программирования и доступного интерфейса является преимуществом.

Стоит заметить, что производительность не будет рассматриваться как критерий, поскольку все рассматриваемые варианты предоставляют функционал достаточный, чтобы реализовать динамическое взаимодействие с объектами в трёхмерной среде.

\section{Анализ возможностей рассмотренных решений}
Ниже в таблице \ref{tab:res_tab} представлены сведения о кроссплатформенности и доступности рассмотренных решений, при этом решения разбиты на два класса --- игровые движки и графические API. Базовым средством решения задачи создания трёхмерной среды в данном случае является решение, функционал которого позволяет в полной мере реализовать трёхмерную виртуальную среду.
\begin{table}[H] 
\captionsetup{justification=raggedright}
\caption{Анализ возможностей рассмотренных решений}
\label{tab:res_tab}
\begin{tabular}{|ccccccc|}
\hline
\multicolumn{1}{|c|}{\multirow{2}{*}{Типы систем}}                                              & \multicolumn{3}{c|}{Игровой движок}                                                          & \multicolumn{3}{c|}{Графический API}                                                  \\ \cline{2-7} 
\multicolumn{1}{|c|}{}                                                                          & \multicolumn{1}{c|}{Unity} & \multicolumn{1}{c|}{Godot} & \multicolumn{1}{c|}{Unreal Engine} & \multicolumn{1}{c|}{Vulkan} & \multicolumn{1}{c|}{DirectX} & Metal                    \\ \hline
\multicolumn{7}{|c|}{Кроссплатформенность}                                                                                                                                                                                                                                             \\ \hline
\multicolumn{1}{|c|}{Windows}                                                                   & \multicolumn{1}{c|}{Да}    & \multicolumn{1}{c|}{Да}    & \multicolumn{1}{c|}{Да}            & \multicolumn{1}{c|}{Да}     & \multicolumn{1}{c|}{Да}      & Нет                      \\ \hline
\multicolumn{1}{|c|}{Mac OS}                                                                    & \multicolumn{1}{c|}{Да}    & \multicolumn{1}{c|}{Да}    & \multicolumn{1}{c|}{Да}            & \multicolumn{1}{c|}{Да}     & \multicolumn{1}{c|}{Нет}     & Да                       \\ \hline
\multicolumn{1}{|c|}{Linux}                                                                     & \multicolumn{1}{c|}{Да}    & \multicolumn{1}{c|}{Да}    & \multicolumn{1}{c|}{Да}            & \multicolumn{1}{c|}{Да}     & \multicolumn{1}{c|}{Нет}     & Нет                      \\ \hline
\multicolumn{7}{|c|}{Стоимость}                                                                                                                                                                                                                                                        \\ \hline
\multicolumn{1}{|c|}{\begin{tabular}[c]{@{}c@{}}Бесплатно\\  без ограничений\end{tabular}}      & \multicolumn{1}{c|}{Нет}   & \multicolumn{1}{c|}{Да}    & \multicolumn{1}{c|}{Нет}           & \multicolumn{1}{c|}{Да}     & \multicolumn{1}{c|}{Да}      & Да                       \\ \hline
\multicolumn{1}{|c|}{\begin{tabular}[c]{@{}c@{}}Бесплатно\\  с ограничениями\end{tabular}}      & \multicolumn{1}{c|}{Да}    & \multicolumn{1}{c|}{Нет}   & \multicolumn{1}{c|}{Да}            & \multicolumn{1}{c|}{Нет}    & \multicolumn{1}{c|}{Нет}     & Нет                      \\ \hline
\multicolumn{7}{|c|}{Доступность}                                                                                                                                                                                                                                                      \\ \hline
\multicolumn{1}{|c|}{Используется C++}                                                          & \multicolumn{1}{l|}{Нет}    & \multicolumn{1}{l|}{Нет}    & \multicolumn{1}{l|}{Да}           & \multicolumn{1}{l|}{Да}     & \multicolumn{1}{l|}{Да}      & \multicolumn{1}{l|}{Да}  \\ \hline
\multicolumn{1}{|c|}{Используется C\#}                                                          & \multicolumn{1}{l|}{Да}   & \multicolumn{1}{l|}{Да}   & \multicolumn{1}{l|}{Нет}            & \multicolumn{1}{l|}{Нет}    & \multicolumn{1}{l|}{Нет}     & \multicolumn{1}{l|}{Нет} \\ \hline
\multicolumn{1}{|c|}{\begin{tabular}[c]{@{}c@{}}Используются\\  собственные языки\end{tabular}} & \multicolumn{1}{l|}{Нет}   & \multicolumn{1}{l|}{Да}    & \multicolumn{1}{l|}{Да}            & \multicolumn{1}{l|}{Нет}    & \multicolumn{1}{l|}{Нет}     & \multicolumn{1}{l|}{Нет} \\ \hline
\multicolumn{7}{|c|}{Необходимость виртуальной среды выполнения}                                                                                                                                                                                                                                                       
 \\ \hline
\multicolumn{1}{|c|}{\begin{tabular}[c]{@{}c@{}}Требуется\\  виртуальная среда\end{tabular}}                                                          & \multicolumn{1}{l|}{Да}    & \multicolumn{1}{l|}{Да}    & \multicolumn{1}{l|}{Нет}           & \multicolumn{1}{l|}{Нет}     & \multicolumn{1}{l|}{Нет}      & \multicolumn{1}{l|}{Нет}  \\ \hline
\end{tabular}
\end{table}

Необходимость включения требования к наличичию виртуальной среды объясняется тем, что некоторые языки (в частности C\#) разрабатываются под специфические программные платформы, являющиеся по сути виртуальной средой выполнения кода. Наличие данной среды позволяет программам, написанных на языках, построенных на основе программной платформы, использовать пространство имён System. Минусом данного подходя является замедление компиляции и работы программы вследствие того, что код, написанный на таком языке, сначала должен быть переведён в байт-код, а затем native-код, который уже может быть выполнен \cite{sharp}.

\section{Анализ дополнительных возможностей реализации программного обеспечения}
Одним из преимуществ игровых движков является то, что они изначально созданы как комплексная система, содержащая в себе множество уже реализованных встроенных подсистем, которые могут быть использованы при реализации трёхмерной виртуальной среды. В частности такими системами являются поддержка многопользовательского режима. Многопользовательский режим является одним из запланированных расширений разработанной трёхмерной среды, и в случае реализации трёхмерной среды при помощи графического API всю систему, которая отвечает за многопользовательскую работу, необходимо будет реализовать с нуля, используя соответствующие библиотеки. Игровые движки с другой стороны предоставляют высокоуровневые API, позволяющие быстро и эффективно реализовать многопользовательский режим работы приложения. 

\section{Вывод из сравнительного анализа существующих решений}
На основе полученного в результате анализа сравнения существующих решений сделаны выводы о том, что в случае, когда в трёхмерной среде необходимо реализовать специфические функции на низком уровне, наилучшим решением будет использование графического API Vulkan, поскольку данный API является кроссплатформенным, распространяется бесплатно, использует C++ как основной язык и не имеет виртуальной среды выполнения. Если необходимо разработать трёхмерную виртуальную среду, затрачивая наименьшее количество человеко-часов, то лучшим решением будет использование Unity, поскольку данный игровой движок является кроссплатформенным, распространяется бесплатно в случае, если прибыль проекта не превышает установленного порога, и использует C\#, являющийся языком программирования более высокого уровня.