\chapter{Конструкторский раздел}

В данном разделе описывается структура программного обеспечения, используемые структуры данных, а также приводятся схемы алгоритмов, используемых при отрисовке изображения и определения наиболее вероятного соответствия объекта, изображённого на двумерном снимке, с одним из объектов из списка тел.

\section{Описание структуры программного обеспечения}
Программное обеспечение состоит из двух доменов --- прикладного, отвечающего за процессы визуализации сцены и распознания трёхмерного объекта, и интерфейсного, отвечающего за взаимодействие с пользователем.

В приложении А на рисунке \ref{fig:classes} приведена диаграмма классов реализуемого программного обеспечения.

% ВСТАВИТЬ СХЕМУ КАКОГО-НИБУДЬ АЛГОРИМТА
\section{Описание используемых алгоритмов}
В данном подразделе рассматриваются основные алгоритмы, использующиеся в разрабатываемом программном обеспечении.
\subsection{Общий алгоритм отрисовки}
Рассмотрим общий алгоритм отрисовки объекта на сцене.
\begin{enumerate}
\item Задать экран, на котором должно быть отображено изображение.
\item Задать объект, который должен быть отображён на экране.
\item Инициализировать матрицу просмотра.
\item Заполнить матрицу просмотра значениями, соответствующими параметрам камеры.
\item Инициализировать z-буфер с размерами, соответствующими размерам экрана.
\item Заполнить элементы z-буфера машинной бесконечностью.
\item Инициализировать вектор растеризируемых треугольных полигонов.
\item Для каждого полигона выполнить пункты 8.1---8.8.
\begin{enumerate}
\item Очистить вектор растеризируемых треугольных полигонов.
\item Рассчитать скалярное произведение нормализованного вектора направления направленного источника света с нормалью полигона.
\item Если скалярное произведение нормализованного вектора направления направленного источника света с нормалью полигона меньше или равно 0, то полигон не закрашивается и происходит переход на пункт . Иначе выполнить пункты 8.3.1---8.3.2.
\begin{enumerate}
\item Инициализировать цвет полигона значением равным скалярному произведению нормализованного вектора направления направленного источника света с нормалью полигона, умноженому на 255 и на интенсивность источника освещения.
\item Если цвет полигона больше 255, то установить величину цвета полигона равной 255.
\end{enumerate}
\item Инициализировать спроецированный полигон и просматриваемый полигон.
\item Перевести полигон в пространство камеры путём умножения векторов вершин полигона на матрицу просмотра.
\item Произвести отсечение невидимой части полигона относительно ближней плоскости пространства отсечения.
\item Для каждого из отсечённых частей полигона выполнить пункты 8.7.1---8.7.2.
\begin{enumerate}
\item Выполнить проецирование полигона на экран.
\item Добавить спроецированный полигон в вектор растеризируемых полигонов.
\end{enumerate}
\item Для каждого полигона из векторов спроецированных полигонов выполнить пункты 8.8.1---8.8.5.
\begin{enumerate}
\item Масштабировать координаты полигона для соответствия размерам экрана.
\item Инициализировать список отсечённых полигонов.
\item Добавить в конец списка отсечённых полигонов рассматриваемый полигон.
\item Выполнить отсечение полигона относительно плоскостей, ограничивающих пирамиду видимости и результат записать в список отсечённых полигонов.
\item Для каждого полигона из списка отсечённых полигонов выполнить закраску с учётом z-буфера.
\end{enumerate}
\end{enumerate}
\item Обновить экран для вывода полученного изображения.
\end{enumerate}

\subsection{Алгоритм отсекания невидимой части треугольного полигона}
Алгоритм отсекания невидимой части треугольного полигона используется в случае, если полигон не находится в пирамиде видимости либо находится внутри её частично. Если не производить отсекание невидимой части, то из-за того, что одна или несколько точек полигона вышли за пределы пирамиды видимости, при рассчёте просматриваемого полигона значения его полей будут уходить в бесконечноть, что в свою очередь приводит аварийному завершению работы программы.

Рассмотрим алгоритм отсекания невидимой части треугольного полигона.
\begin{enumerate}
\item Задать точку на плоскости, относительно которой проводится отсечение.
\item Задать нормаль к плоскости, относительно которой проводится отсечение.
\item Задать полигон, подлежащий отсечению.
\item Задать первую часть отсекаемого полигона.
\item Задать вторую часть отсекаемого полигона.
\item Задать нормаль к полигону.
\item Инициализировать вектор вершин, находящихся внутри пирамиды видимости.
\item Инициализировать вектор вершин, находящихся вне пирамиды видимости.
\item Инициализировать количество точек на внутренней стороне плоскости.
\item Инициализировать количество точек на внешней стороне плоскости.
\item Для каждой вершины полигона выполнить пункты 11.1---11.2.
\begin{enumerate}
\item Вычислить расстояние от вершины до плоскости, относительно которой проводится отсечение.
\item Если расстояние больше или равно нулю, то добавить вершину в вектор вершин, находящихся внутри пирамиды видимости. Иначе добавить вершину в вектор вершин, находящихся вне пирамиды видимости.
\end{enumerate}
\item Если количество вершин внутри пирамиды видимости равно нулю, вернуть ноль. Иначе продолжить.
\item Если количество вершин внутри пирамины видимости равно трём, то выполнить пункты 13.1---13.3. Иначе перейти на пункт 14.
\begin{enumerate}
\item Скопировать значения полигона, подлежащего отсечению, в поля первого отсекаемого полигона.
\item Если скалярное произведение нормали полигона, подлежащего отсечению на нормаль первого отсекаемого полигона меньше нуля, то изменить порядок вершин полигона, подлежащего отсечению так, чтобы удовлетворять общему порядку обхода вершин.
\item Вернуть один.
\end{enumerate}
\item Если количество вершин внутри пирамиды видимости равно одному и количество вершин вне пирамиды видимости равно двум, то выполнить пункты 14.1---14.3. Иначе перейти на пункт 15.
\begin{enumerate}
\item Разделить полигон плоскостью отсечения и часть, находящуюся внутри пирамиды видимости занести в первую часть отсекаемого полигона. 
\item Если скалярное произведение нормали полигона, подлежащего отсечению на нормаль первого отсекаемого полигона меньше нуля, то изменить порядок вершин отсечённого полигона, чтобы удовлетворять общему порядку обхода вершин.
\item Вернуть один.
\end{enumerate}
\item Если количество вершин внутри пирамиды видимости равно двум и количество вершин вне пирамиды видимости равно одному, то выполнить пункты 15.1---15.5. Иначе перейти на пункт 16.
\begin{enumerate}
\item Разделить полигон плоскостью отсечения.
\item Четырёхугольник, находящийся внутри пирамиды видимости, разделить на два треугольных полигона и занести их в первую часть отсекаемого полигона и вторую часть отсекаемого полигона.
\item Если скалярное произведение нормали полигона, подлежащего отсечению на нормаль первого отсекаемого полигона меньше нуля, то изменить порядок вершин отсечённого полигона, чтобы удовлетворять общему порядку обхода вершин.
\item Если скалярное произведение нормали полигона, подлежащего отсечению на нормаль второго отсекаемого полигона меньше нуля, то изменить порядок вершин отсечённого полигона, чтобы удовлетворять общему порядку обхода вершин.
\item Вернуть два.
\end{enumerate}
\item Вернуть ноль.
\end{enumerate}

\subsection{Алгоритм z-буфера}
Алгоритм z-буфера приводится в виде схемы в приложении В на рисунках \ref{fig:z_buf_1}---\ref{fig:z_buf_2}.

\section{Описание архитектуры используемой свёрточной нейросети}
Использующаяся в работе нейросеть состоит из тридцати одного слоя. Состоящего из пяти блоков свёртки и одного блока скрытых слоёв. Блоки свёртки имеют увеличивающуюся глубину слоя, от 16 в первом блоке до 128 в последнем, и уменьшающийся размер ядра свёртки, от 9 до 3. Данная структура позволяет нейросети в первых слоях свёртки распознавать более глобальные признаки и постепенно переходить к распознаванию локальных признаков. Результат последнего блока свёртки трансформируется в одномерный вектор и передаётся на вход скрытым слоям. Результатом работы модели является индекс трёхмерного объекта, наиболее подходящего объекту, изображённому на двумерном изображении.

В приложении Б на рисунке \ref{fig:cnn} приведена схема архитектуты разработанной нейросети.

\subsection{Алгоритм обучения свёрточной нейросети}
Рассмотрим алгоритм обучения свёрточной нейросети.
\begin{enumerate}
\item Задать набор изображений для обучения.
\item Преобразовать каждое изображение из набора так, чтобы оно подходило по формату к входному слою модели.
\item Задать количество эпох обучения и параметры оптимизатора модели.
\item Скомпилировать модель.
\item Обучить модель на предоставленных данных.
\item Сохранить результат в файл модели.
\end{enumerate}

\subsection{Алгоритм определения трёхмерного объекта, наиболее вероятно являющимся объектом изображённым на снимке}
Рассмотрим алгоритм определения трёхмерного объекта, наиболее вероятно являющимся объектом изображённым на снимке.
\begin{enumerate}
\item Задать снимок.
\item Изменить формат снимка на градации серого.
\item Обрезать изображение по минимальному размеру до квадратного относительно центра.
\item Изменить размер изображения на 256 на 256 пикселей.
\item Загрузить модель нейросети из файла.
\item Для каждого из трёхмерных объектов из списка объектов вычислить вероятность того, что данный объект изображён на поданном на вход изображении.
\item Вернуть индекс трёхмерного объекта, вероятность которого является максимальной из всех рассчитаных.
\end{enumerate}

\section{Описание используемых структур данных}
В данном подразделе рассматриваются структуры данных, необходимые для решения задачи.

Трёхмерный вектор --- эта структура данных использующется при задании положения точки в трёхмерном пространстве. Задаётся вещественными значениями координат точки $x$, $y$ и $z$.

Треугольный полигон --- эта структура данных используется при задании трёхмерного объекта, позволяя описать поверхность как совокупность треугольных полигонов. Выбор треугольного полигона обусловлен тем, что три точки в пространстве однозначно задают плоскость. Задаётся как список из трёх трёхмерных векторов, описывающих положения вершин треугольного полигона.

Матрица --- эта стурктура данных используется для хранения матриц коэффициентов и рассчёта просматриваемой части трёхмерного объекта и проекции трёхмерного объекта на экран. Задаётся как двумерный массив вещественных чисел.

Трёхмерный объект --- это структура данных используется для описания трёхмерных объектов в пространстве сцены. Описывает основные параметры трёхмерных объектов, включающие в себя положение объекта в пространстве сцены, цвет закраски объекта и описание поверхности объекта.

Камера --- эта структура данных описывает виртуальную камеру. Используется при синтезе изображения сцены. Описывает ориентацию камеры в пространстве и матрицу просмотра, соответствующую этой камере.

Освещение --- эта структура данных используется для задания источника овещения. Описывает направление распространения, цвет и интенсивность освещения.

Сцена --- эта структура данных используется для задания трёхмерной сцены. Описывает множество находящихся на сцене трёхмерных объектов, камеру и источник освещения.

\section{Вывод из конструкторского раздела}
В данном разделе была рассмотрена структура разрабатываемого программного обеспечения, представлена диаграмма классов и описаны основные алгоритмы, использующиеся в разрабатываемом программном обеспечении. Были определены классы эквивалентности и тесты для каждого из описанных алгоритмов, и описаны используемые структуры данных.