\Conclusion
В процессе выполнения данной лабораторной работы были изучены алгоритм полного перебора и муравьиный алгоритм. Были выполнены анализ алгоритмов и представлены схемы алгоритмов, а также функциональная схема ПО. После чего эти алгоритмы были реализованы при помощи языка C++ в IDE Visual Studio 2019. Помимо этого были произведены эксперименты с целью получить информацию о временной производительности алгоритмов. В результате эксперимента было получено, что оптимальными параметрами для муравьиного алгоритма являются количество итераций - 300, количество муравьёв - 5, альфа - 0.6, бета - 0.4, уровень испарения - 0.8, количество ферромона для муравья - 10. При сравнении времени выполнения при достижении 10 узлов муравьиный алгоритм работает более чем в 27 раз быстрее. В результате можно сделать вывод о том, что использование муравьиного алгоритма позволяет решать задачу коммивояжера значительно быстрее, чем алгоритм полного перебора. Целью данной лабораторной работы являлось изучение алгоритмов полного перебора и муравьиного алгоритма на примере задачи коммивояжера, что было успешно достигнуто.