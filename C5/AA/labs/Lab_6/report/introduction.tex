\Introduction
Задача коммивояжера — одна из самых известных задач комбинаторной оптимизации, заключающаяся в поиске самого выгодного маршрута, проходящего через указанный города хотя бы по одному разу с последующим возвратом в исходный город. [1] Стоит заметить, что использование методов вычисления, которые гарантированно находят самый лучший результат, например метод полного перебора всех путей, приводит к тому, что решение задачи таким способом невозможно в рамках текущих мощностей вычислительных машин. Поэтому для решения этой задачи используют эврестические алгоритмы, которые не могут дать абсолютно верного решения, однако позволяют получить результаты достаточно близкие к нему. 

Муравьиный алгоритм — один из эффективных полиномиальных алгоритмов для нахождения приближенных решений задачи коммивояжера, а также решения аналогичных задач поиска маршрутов на графах. Суть подхода заключается в анализе и использовании модели поведения муравьев, ищущих пути от колонии к источнику питания, и представляет собой метаэвристическую оптимизацию. [2]

Целью данной работы является изучение муравьиного алгоритма на материале решения задачи коммивояжера. Для того, чтобы достичь поставленной цели, нам необходимо выполнить следующие задачи:

\begin{enumerate}
	\item провести анализ алгоритмов полного перебора и муравьиного алгоритма;
	\item описать используемые структуры данных;
	\item привести схемы рассматриваемых алгоритмов;
	\item программно реализовать описанные выше алгоритмы;
	\item методом подбора найти оптимальные параметры для муравьиного алгоритма;
	\item провести сравнительный анализ каждого алгоритма по затрачиваемому в процессе работы времени.
\end{enumerate}