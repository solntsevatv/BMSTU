\Conclusion
В процессе выполнения данной лабораторной работы были изучены алгоритмы нахождения расстояния Левенштейна (нерекурсивный, рекурсивный и рекурсивный с заполнением и дополнительными проверками матрицы) и алгоритм нахождения Дамерау-Левенштейна. Было выполнено сравнение по памяти выше перечисленных алгоритмов, в ходе которого был сделан вывод, что нерекурсивный алгоритм нахождения расстояния Левенштейна использует наименьшее количество памяти. В результате эксперимента было получено, что на рандомно генерирующейся строке длиной от 1 до 9 с шагом 1, наименее времяёмким из всех рассмотренных алгоритмов является табличный алгоритм Левенштейна. В результате можно сделать вывод, что для строк длинной от 1 до 9 символов данных предпочтительно использовать табличный алгоритм Левенштейна. Также были изучены зависимости времени выполнения алгоритмов от длины строк и были выполнены все поставленные задачи. Целью данной лабораторной работы являлось изучения алгоритмов нахождения минимального редакторского расстояния, что было успешно достигнуто. 