\Introduction
Расстояние Левенштейна (редакционное расстояние) - минимальное количество односимвольных операций (вставки, удаления, замены), необходимых для преобразования одной последовательности символов в другую. 
Расстояние Левенштейна используется в решении таких задач, как:
\begin{itemize}
	\item исправление ошибок в слове;
	\item сравнение генов, хромосом и белков в биоинформатике;
	\item сравнение текстовых файлов утилитой diff и ей подобными.
\end{itemize}

Целью данной работы является изучение, реализация и сравнительный анализ различных алгоритмов нахождения расстояния Левенштейна и Дамерау-Левенштейна, а также получение практических навыков при реализации указанных алгоритмов. Для достижения поставленной цели необходимо выполнить следующие задачи:
\begin{enumerate} 
	\item изучить алгоритмы:
	\begin{enumerate} 
		\item поиска расстояния Левенштейна с помощью табличного способа;
		\item поиска расстояния Левенштейна с помощью рекурсивного способа;
		\item поиска расстояния Левенштейна с помощью рекурсивного способа с заполнением и дополнительными проверками матрицы;
		\item рпоиска расстояния Дамерау-Левенштейна с помощью табличного способа.
	\end{enumerate}
	\item построить схемы алгоритмов и провести функциональный анализ ПО;
	\item сделать сравнительный анализ данных алгоритмов по затрачиваемым в процессе работы алгоритмов ресурсам (памяти и времени);
	\item получить экспериментальное подтверждение различий эффективности различных версий реализации данных алгоритмов (рекурсивной и нерекурсивной) при помощи программного обеспечения на основе замеров процессорного времени выполнения реализации данных алгоритмов для строк различной длины;
	\item сформировать описание и обоснование полученных в процессе работы результатов в отчёте о выполненной лабораторной работе, выполненного как расчётно-пояснительная записка к работе.
\end{enumerate}