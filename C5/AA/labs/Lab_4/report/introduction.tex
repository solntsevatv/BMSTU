\Introduction
Параллельное программирование - способ организации компьютерных вычислений, при котором программы разрабатываются как набор взаимодействующих вычислительных процессов, работающих параллельно (одновременно). Поскольку ресурсы компьютера ограничены, параллелизация алгоритмов и последующая их реализация в параллельном виде позволит значительно ускорить их выполнение. В данной лабораторной работе мы будем рассматривать параллельное программирование на примере параллелизации алгоритма свёртки.

Целью данной лабораторной работы является изучение изучение и реализация параллельного программирования в рамках решения задачи параллельной и непараллельной реализации алгоритма свёртки. Для того, чтобы достичь поставленной цели, нам необходимо выполнить следующие задачи:

\begin{enumerate}
	\item провести анализ алгоритма свёртки;
	\item описать используемые структуры данных;
	\item привести схемы рассматриваемого алгоритма для параллельной и непараллельной реализации;
	\item программно реализовать данные выше алгоритмы;
	\item провести сравнительный анализ каждого алгоритма по затрачиваемому в процессе работы времени в зависимости от количества потоков.
\end{enumerate}
