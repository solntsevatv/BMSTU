\chapter*{Заключение}
\addcontentsline{toc}{chapter}{Заключение}
В рамках курсового проекта реализована программа, повзоляющая загружать трёхмерную модель, редактировать масштаб, вращение и положение трёхмерной модели, выводить на экран и сохранять изображение трёхмерной модели, и определять соответствие объекта, изображённого на двумерном снимке, с одним из объектов из списка тел. 

В ходе выполнения проекта были выполнены следующие задачи:
\begin{itemize}
	\item исследованы подходы к синтезу изображений;
	\item исследованы способы создания двумерного снимка;
	\item исследованы алгоритмы нахождения трёхмерного объекта по двумерному снимку;
	\item описаны используемые при разработке ПО алгоритмы;
	\item определены средства программной реализации;
	\item реализованы алгоритмы отрисовки сцены;
	\item реализован алгоритм позволяющий определять наиболее вероятное соответствие объекта, изображённого на двумерном снимке, с одним из объектов из списка тел;
	\item исследована зависимость скорости алгоритма отрисовки от количества полигонов в объекте;
	\item исследована зависимость скорости алгоритма отрисовки от размера проекции полигона;
	\item исследовна эффективность работы алгоритма алгоритма определения наиболее вероятного соответствия объекта на изображении объекту из списка тел.
\end{itemize}

В результате эксперимента по оценке зависимости скорости работы алгоритма отрисовки от количества полигонов в просматриваемой модели было выявлено, что скорость работы алгоритма отрисовки зависит от количества полигонов так, что чем больше полигонов, тем быстрее работает алгоритм отрисовки, достигая разницы более чем в 1.6 раз.

В результате эксперимента по оценке зависимости скорости работы алгоритма отрисовки от размера полигонов было выявлено, что скорость работы алгоритма отрисовки зависит от количества полигонов так, что чем больше размер проекции полигона на экран, тем медленнее работает алгоритм, достигая разницы в более чем в 7.5 раз.

В результате эксперимента по оценке эффективности работы разработанного алгоритма определения трёхмерного объекта, наиболее вероятно являющимся объектом изображённым на снимке было выявлено, что алгоритм корректно распознаёт объекты, имеющие заметно отличающуюся внешнюю форму, такие как тор, конус, циллиндр, изосфера и сфера. Однако если объекты имеют похожую геометрию, то алгоритм не во всех случаях может определить корректную модель, что происходит в случае куба и пирамиды.