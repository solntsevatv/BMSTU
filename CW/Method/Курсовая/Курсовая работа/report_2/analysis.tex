\chapter{Анализ предметной области}
Реализация трёхмерной среды подразумевает создание приложения, с помощью которого можно проводить эксперименты. Так как создание программного обеспечения с нуля может оказаться более затратным, чем использование каких-либо готовых инструментов для создания приложений, необходимо рассмотреть данные инструменты и оценить, насколько выгодным является их использование в данном проекте.

Перед тем, как начать рассмотрение данных инструментов и технологий, необходимо ввести набор критериев, который будет является определяющим при объективном выборе того или иного инструмента для создания трёхмерной среды для визуализации окружения робота Ф-2. Критерии в свою очередь исходят от требований к разрабатываемому программному обеспечению. В набор данных критериев входят:
\begin{itemize}
\item кроссплатформенность --- при поддержке трёхмерной визуальной средой различных платформ (персональный компьютер, смартфон) и различных операционных систем (Windows, Mac OS, Linux) проведение экспериментов становится более доступным, так как не требуется устанавливать виртуальные машины с нужной системой или покупать новое устройство, которое совместимо с трёхмерной средой;
\item наличие инструментов для разработки трёхмерной среды --- базовое требование необходимое для решения задачи создания трёхмерной среды.
\end{itemize}

\section{Игровые движки}

Игровой движок --- это программная среда, предназначенная в первую очередь для разработки видеоигр и обычно включающая в себя соответствующие библиотеки и вспомогательные программы. \cite{gmengine_opr} В нашем случае использование игрового движка может позволит снизить время разработки, и обеспечить разработчиков набором уже написанных базовых инструментов, которые можно использовать в процессе разработке виртуальной среды. Использование игрового движка имеет следующие недостатки:
\begin{itemize}
\item зависимость от игрового движка --- наиболее очевидная проблема, возникающая вследствие того, что в большей части случаев код игрового движка является закрытым, в результате чего внутренние проблемы движка не могут быть решены разработчиком приложения, так как доступа к внутреннему коду нет;
\item стоимость использования игрового движка --- большая часть игровых движков имеет бесплатную версию в которой недоступны многие инструменты, которые являются необходимыми для разработки и поддержки приложения.
\end{itemize}

На основе требований к разрабатываемой трёхмерной визуальной среде и особенностей использования игровых движков сформируем требования, с помощью которых будет произведен отбор рассматриваемых ниже движков. В набор входят следующие требования:
\begin{itemize}
\item наличие инструментов для разработки трёхмерной среды --- движок должен поддерживать работу с трёхмерными объектами и иметь встроенный набор инструментов для работы с ними, данный критерий является обязательным;
\item производительность --- движок должен иметь либо встроенные средства оптимизации, либо возможность их имплементации для ускорения работы приложения;
\item стоимость использования --- вследствие особенностей разработки приложения игровой движок должен быть либо бесплатным, либо предоставлять возможность бесплатно использовать движок в некоммерческих целях;
\item кроссплатформенность --- движок должен позволять разрабатывать приложения под большинство наиболее популярных платформ и операционных систем (Windows, Mac OS, Linux);
\item доступность --- использование движком нестандартных языков программирования и систем, не являющихся общепринятым стандартом в своей области может сильно осложнить разработку трёхмерной визуальной среды, поэтому использование популярных языков программирования и доступного интерфейса является преимущетсвом.
\end{itemize}

Согласно статистике, собранной платформой itch.io \cite{itch}, в состав наиболее популярных игровых движков (имеющих наибольшее количество проектов) входят следующие движки:
\begin{itemize}
\item Unity --- 82 тысячи проектов, прирост за неделю --- 349;
\item Construct --- 18.3 тысячи проектов, прирост за неделю --- 156;
\item GameMaker:Studio --- 12.1 тысячи проектов, прирост за неделю --- 56;
\item Godot --- 8.640 тысяч проектов, прирост за неделю --- 82;
\item Twine --- 7.872 тысячи проектов, прирост за неделю --- 51;
\item Unreal Engine --- 4.819 тысяч проектов, прирост за неделю --- 27;
\item Bitsy --- 4.814 тысяч проектов, прирост за неделю --- 27;
\item RPG Maker --- 4.531 тысяч проектов, прирост за неделю --- 17;
\item PICO-8 --- 4.234 projects тысяч проектов, прирост за неделю --- 19.\\
\end{itemize}

Остальные движки имеют менее четырёх тысяч проектов и гораздо меньший прирост по количеству, поэтому рассматриваться не будут. Также необходимо изначально учесть первый критерий --- наличие инструментов для работы с трёхмерными объектами.
\begin{itemize}
\item Unity --- данный движок имеет встроенную поддержку трёхмерной графики \cite{unity}.
\item Construct --- данный движок имеет только инструменты для 2D разработки, поэтому он не может быть использован при разработке трёхмерной визуальной среды \cite{construct}.
\item GameMaker:Studio --- данный движок имеет ограниченную поддержку трёхмерной разработки, включающую основные трансформации трёхмерных объектов и возможность создания нескольких примитивов. Этого достаточно для того, чтобы разработать прототип проекта, однако возможности развития проекта при использовании данного движка органичены \cite{gamemaker}.
\item Godot --- данный движок имеет встроенную поддержку трёхмерной графики \cite{godot}.
\item Twine --- данный движок используется для разработки интерактивных текстовых игр и поэтому для нашей задачи не подходит \cite{twine}.
\item Unreal Engine --- данный движок имеет встроенную поддержку трёхмерной графики \cite{unreal}.
\item Bitsy --- не имеет поддержки трёхмерной графики \cite{bisty}.
\item RPG Maker --- не имеет поддержки трёхмерной графики \cite{rpg}.
\end{itemize}

В результате из рассматриваемых движков остаются следующие: Unity, Godot, Unreal Engine. Рассмотрим их подробнее.

\subsection{Технологии}
\subsubsection{Unity}
Unity поддерживает графические API DirectX, Metal, OpenGL и Vulkan в зависимости от доступности API на конкретной платформе. Unity использует встроенный набор графических API или графические API, которые выбираются в редакторе \cite{unity}.
\subsubsection{Unreal Engine}
Система рендеринга в Unreal Engine использует конвейеры, основанные на технологиях DirectX 11 и DirectX 12. Она включает в себя отложенное затенение, глобальное освещение, освещение прозрачных объектов и пост-обработку, также как и симуляциию частиц на графическом процессоре с использованием векторных полей \cite{unreal}.
\subsubsection{Godot}
Godot использует OpenGL ES 3.0 для высококачественного рендеринга (и OpenGL 3.3 на персональных компьютерах). Это обеспечивает совместимость со всеми настольными ПК, мобильными устройствами и WebGL 2 \cite{godot}.

\subsection{Стоимость использования движка}
\subsubsection{Unity}
Лицензирование Unity делится на 4 уровня.
\begin{itemize}
\item Personal --- если доход от использования приложения не превышает 100 000 долларов, то для разработчика доступна поддержка 20 пользователей, одновременно использующих приложения, стандартная сборка в облачном хранилище. Стоимость --- бесплатно.
\item Plus --- если доход от использования приложения не превышает 100 000 долларов, то для разработчика доступна поддержка 50 пользователей, одновременно использующих приложения, приоритетная сборка в облачном хранилище, отчёты по производительности. Стоимость --- 399 долларов в год.
\item Pro --- если доход от использования приложения не превышает 100 000 долларов, то для разработчика доступна поддержка 200 пользователей, одновременно использующих приложения, одновременная сборка в облачном хранилище, отчёты по производительности, премиум поддержка. Стоимость --- 1800 долларов в год.
\item Enterprise --- если доход от использования приложения не превышает 100 000 долларов, то для разработчика доступна поддержка пользовательского мультиплеера, выделенные ресурсы в облачном хранилище, отчёты по производительности, премиум поддержка и доступ к исходному коду. Стоимость - 200 долларов в месяц \cite{unity}.
\end{itemize}
\subsubsection{Unreal Engine}
Unreal Engine 4 является бесплатным, однако разработчики должны передавать 5\% от выручки с продаж приложения компании Epic Games, если выручка превышает 1 000 000 долларов \cite{unreal}. 
\subsubsection{Godot}
Godot Engine --- это бесплатное программное обеспечение с открытым исходным кодом, выпущенное под разрешительной лицензией MIT (также называемой лицензией Expat). Эта лицензия предоставляет пользователям ряд свобод:
\begin{itemize}
\item можно использовать Godot Engine для любых целей;
\item можно изучить, как работает Godot Engine, и изменить его;
\item можно распространять немодифицированные и измененные версии Godot Engine даже на коммерческой основе и под другой лицензией (включая проприетарную).
\end{itemize}

Единственное ограничение этой третьей свободы заключается в том, что вам необходимо распространять уведомление об авторских правах и заявление о лицензии Godot Engine всякий раз, когда вы его распространяете. Таким образом, ваш производный продукт может иметь другую лицензию, но в документации должно быть указано, что он является производным от движка Godot Engine, лицензированного MIT \cite{godot}.

\subsection{Кроссплатформенность}
\subsubsection{Unity}
Unity поддерживает следующие ОС:
\begin{itemize}
\item мобильные ОС --- iOS, Android;
\item не мобильные ОС --- Windows, Maс, Linux \cite{unity_comp}.
\end{itemize}
\subsubsection{Unreal Engine}
Unity поддерживает следующие платформы:
\begin{itemize}
\item мобильные ОС --- iOS, Android;
\item не мобильные ОС --- Windows, Maс, Linux \cite{unreal_comp}.
\end{itemize}
\subsubsection{Godot}
Unity поддерживает следующие платформы:
\begin{itemize}
\item мобильные ОС --- iOS, Android;
\item не мобильные ОС --- Windows, Maс, Linux \cite{godot_comp}.
\end{itemize}

Стоит заметить, что все три рассматриваемых движка поддерживают серию различных консольных платформ и различные VR-платформы, однако в нашем случае мы это не учитываем, поскольку они не относятся к целевым платформам разрабатываемого приложения.

\subsection{Язык программирования для разработки}
\subsubsection{Unity}
При разработке приложений на Unity используется код, написанный на языке C\# \cite{unity}.
\subsubsection{Unreal Engine}
При разработке приложений на Unreal Engine используется код написанный либо на C++, либо на Blueprint --- язык разработанный специально для Unreal Engine \cite{unreal}.
\subsubsection{Godot}
При разработке приложений на Godot используется код, написанный на GDScript, C\#, GDVisual \cite{godot}.

\section{Программные интерфейсы}
API (программный интерфейс приложения) --- описание способов (набор классов, процедур, функций, структур или констант), которыми одна компьютерная программа может взаимодействовать с другой программой. Обычно входит в описание какого-либо интернет-протокола, программного каркаса (фреймворка) или стандарта вызовов функций операционной системы \cite{linux}.

\subsection{Графические API}
На сегодняшний день существует множество графических API, которые могут быть использованы для реализации графической части проекта. Рассмотрим наиболее популярные из них на сегодняшний день: OpenGL, Vulkan, WebGL, DirectX, OpenGL ES, WebGPU, Metal, Mantle \cite{graph_engine}.\\

Составим список требований, которым должно удовлетворять графическое API:
\begin{itemize}
\item кроссплатформенность --- графический API должен поддерживать целевые платформы, для которых разрабатывается приложение;
\item производительность --- графический API должен обеспечивать достаточную скорость рендеринга изображения для обеспечения работы приложения в реальном времени;
\item доступность --- использование графическим API нестандартных языков программирования и систем, не являющихся общепринятым стандартом в своей области, может сильно осложнить разработку трёхмерной визуальной среды, поэтому использование популярных языков программирования и доступного интерфейса является преимущетсвом;
\item стоимость использования --- вследствие особенностей разработки приложения игровой движок должен быть либо бесплатным, либо предоставлять возможность бесплатно использовать движок в некоммерческих целях.
\end{itemize}

\subsubsection{Кросплатформенность}
Рассмотрим платформы, поддерживаемые представленными выше графическими API:
\begin{itemize}
\item OpenGL --- Windows, Linux, Mac, FreeBSD;
\item Vulkan --- Windows, Linux, Max, FreeBSD;
\item WebGL --- Web platform;
\item DirectX --- Windows;
\item OpenGL ES --- Linux, Windows;
\item Metal --- iOS, Mac, tvOS.
\end{itemize}

Целевыми платформами данного проекта являются Windows и Linux, поэтому из всех предложенных графических API выбраны OpenGL, Vulkan, DirectX, OpenGL ES, Metal, Mantle. Остальные рассмотрены не будут. Также стоит заметить, что согласно официальному сайте OpenGL, развитие данного графического API прекращено в пользу развития Vulkan, поэтому использование OpenGL и OpenGL ES подразумевает риск для возможности дальнейшей поддержки разработанного приложения. Вследствие чего они не будут включаться в рассматриваемые графические API.

\subsubsection{Стоимость}
Рассмотрим стоимость использования графических API:
\begin{itemize}
\item Vulkan --- распространяется бесплатно \cite{vulkan};
\item DirectX --- распространяется бесплатно\cite{direct};
\item Metal --- распространяется бесплатно \cite{metal}.
\end{itemize}

\subsubsection{Доступность}
Рассмотрим доступность графических API:
\begin{itemize}
\item Vulkan --- основным способом работы с Vulkan является использование официальной библиотеки под C++, однако помимо неё существует серия неофициальных библиотек, позволяющих использовать такие языки, как Python и C\# \cite{vulkan};
\item DirectX --- основным способом работы с DirectX является использование официальной библиотеки от Microsoft под C++ \cite{direct};
\item Metal --- для работы с Metal API используется официальный интерфейс от Apple, написанный на C++ \cite{metal}.
\end{itemize}

\subsubsection{Производительность}
Согласно данным, полученным в ходе экспериментов на серии различных видеокарт, Vulkan работает быстрее DirectX в среднем на 5 \% \cite{prod}. Если сравнивать Vulkan и Metal, то согласно данным, полученным в ходе экспериментов на серии различных видеокарт, Metal работает быстрее Vulkan на графических процессорах AMD, достигая преимущества от 8 до 40 процентов в производительности. Однако при тестировании на картах от NVIDIA Vulkan выигрывает от 5 до 30 процентов в производительности.

\section{Вывод из анализа предметной области}
В результате проведённого анализа были рассмотрены решения, позволяющие решить задачу создания трёхмерной виртуальной среды.